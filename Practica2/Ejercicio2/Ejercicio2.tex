\documentclass[]{article}
\usepackage{lmodern}
\usepackage{amssymb,amsmath}
\usepackage{ifxetex,ifluatex}
\usepackage{fixltx2e} % provides \textsubscript
\ifnum 0\ifxetex 1\fi\ifluatex 1\fi=0 % if pdftex
  \usepackage[T1]{fontenc}
  \usepackage[utf8]{inputenc}
\else % if luatex or xelatex
  \ifxetex
    \usepackage{mathspec}
  \else
    \usepackage{fontspec}
  \fi
  \defaultfontfeatures{Ligatures=TeX,Scale=MatchLowercase}
\fi
% use upquote if available, for straight quotes in verbatim environments
\IfFileExists{upquote.sty}{\usepackage{upquote}}{}
% use microtype if available
\IfFileExists{microtype.sty}{%
\usepackage{microtype}
\UseMicrotypeSet[protrusion]{basicmath} % disable protrusion for tt fonts
}{}
\usepackage[margin=1in]{geometry}
\usepackage{hyperref}
\hypersetup{unicode=true,
            pdftitle={Laboratorio 2},
            pdfauthor={Avila Santos Alex},
            pdfborder={0 0 0},
            breaklinks=true}
\urlstyle{same}  % don't use monospace font for urls
\usepackage{color}
\usepackage{fancyvrb}
\newcommand{\VerbBar}{|}
\newcommand{\VERB}{\Verb[commandchars=\\\{\}]}
\DefineVerbatimEnvironment{Highlighting}{Verbatim}{commandchars=\\\{\}}
% Add ',fontsize=\small' for more characters per line
\usepackage{framed}
\definecolor{shadecolor}{RGB}{248,248,248}
\newenvironment{Shaded}{\begin{snugshade}}{\end{snugshade}}
\newcommand{\KeywordTok}[1]{\textcolor[rgb]{0.13,0.29,0.53}{\textbf{#1}}}
\newcommand{\DataTypeTok}[1]{\textcolor[rgb]{0.13,0.29,0.53}{#1}}
\newcommand{\DecValTok}[1]{\textcolor[rgb]{0.00,0.00,0.81}{#1}}
\newcommand{\BaseNTok}[1]{\textcolor[rgb]{0.00,0.00,0.81}{#1}}
\newcommand{\FloatTok}[1]{\textcolor[rgb]{0.00,0.00,0.81}{#1}}
\newcommand{\ConstantTok}[1]{\textcolor[rgb]{0.00,0.00,0.00}{#1}}
\newcommand{\CharTok}[1]{\textcolor[rgb]{0.31,0.60,0.02}{#1}}
\newcommand{\SpecialCharTok}[1]{\textcolor[rgb]{0.00,0.00,0.00}{#1}}
\newcommand{\StringTok}[1]{\textcolor[rgb]{0.31,0.60,0.02}{#1}}
\newcommand{\VerbatimStringTok}[1]{\textcolor[rgb]{0.31,0.60,0.02}{#1}}
\newcommand{\SpecialStringTok}[1]{\textcolor[rgb]{0.31,0.60,0.02}{#1}}
\newcommand{\ImportTok}[1]{#1}
\newcommand{\CommentTok}[1]{\textcolor[rgb]{0.56,0.35,0.01}{\textit{#1}}}
\newcommand{\DocumentationTok}[1]{\textcolor[rgb]{0.56,0.35,0.01}{\textbf{\textit{#1}}}}
\newcommand{\AnnotationTok}[1]{\textcolor[rgb]{0.56,0.35,0.01}{\textbf{\textit{#1}}}}
\newcommand{\CommentVarTok}[1]{\textcolor[rgb]{0.56,0.35,0.01}{\textbf{\textit{#1}}}}
\newcommand{\OtherTok}[1]{\textcolor[rgb]{0.56,0.35,0.01}{#1}}
\newcommand{\FunctionTok}[1]{\textcolor[rgb]{0.00,0.00,0.00}{#1}}
\newcommand{\VariableTok}[1]{\textcolor[rgb]{0.00,0.00,0.00}{#1}}
\newcommand{\ControlFlowTok}[1]{\textcolor[rgb]{0.13,0.29,0.53}{\textbf{#1}}}
\newcommand{\OperatorTok}[1]{\textcolor[rgb]{0.81,0.36,0.00}{\textbf{#1}}}
\newcommand{\BuiltInTok}[1]{#1}
\newcommand{\ExtensionTok}[1]{#1}
\newcommand{\PreprocessorTok}[1]{\textcolor[rgb]{0.56,0.35,0.01}{\textit{#1}}}
\newcommand{\AttributeTok}[1]{\textcolor[rgb]{0.77,0.63,0.00}{#1}}
\newcommand{\RegionMarkerTok}[1]{#1}
\newcommand{\InformationTok}[1]{\textcolor[rgb]{0.56,0.35,0.01}{\textbf{\textit{#1}}}}
\newcommand{\WarningTok}[1]{\textcolor[rgb]{0.56,0.35,0.01}{\textbf{\textit{#1}}}}
\newcommand{\AlertTok}[1]{\textcolor[rgb]{0.94,0.16,0.16}{#1}}
\newcommand{\ErrorTok}[1]{\textcolor[rgb]{0.64,0.00,0.00}{\textbf{#1}}}
\newcommand{\NormalTok}[1]{#1}
\usepackage{graphicx,grffile}
\makeatletter
\def\maxwidth{\ifdim\Gin@nat@width>\linewidth\linewidth\else\Gin@nat@width\fi}
\def\maxheight{\ifdim\Gin@nat@height>\textheight\textheight\else\Gin@nat@height\fi}
\makeatother
% Scale images if necessary, so that they will not overflow the page
% margins by default, and it is still possible to overwrite the defaults
% using explicit options in \includegraphics[width, height, ...]{}
\setkeys{Gin}{width=\maxwidth,height=\maxheight,keepaspectratio}
\IfFileExists{parskip.sty}{%
\usepackage{parskip}
}{% else
\setlength{\parindent}{0pt}
\setlength{\parskip}{6pt plus 2pt minus 1pt}
}
\setlength{\emergencystretch}{3em}  % prevent overfull lines
\providecommand{\tightlist}{%
  \setlength{\itemsep}{0pt}\setlength{\parskip}{0pt}}
\setcounter{secnumdepth}{0}
% Redefines (sub)paragraphs to behave more like sections
\ifx\paragraph\undefined\else
\let\oldparagraph\paragraph
\renewcommand{\paragraph}[1]{\oldparagraph{#1}\mbox{}}
\fi
\ifx\subparagraph\undefined\else
\let\oldsubparagraph\subparagraph
\renewcommand{\subparagraph}[1]{\oldsubparagraph{#1}\mbox{}}
\fi

%%% Use protect on footnotes to avoid problems with footnotes in titles
\let\rmarkdownfootnote\footnote%
\def\footnote{\protect\rmarkdownfootnote}

%%% Change title format to be more compact
\usepackage{titling}

% Create subtitle command for use in maketitle
\newcommand{\subtitle}[1]{
  \posttitle{
    \begin{center}\large#1\end{center}
    }
}

\setlength{\droptitle}{-2em}
  \title{Laboratorio 2}
  \pretitle{\vspace{\droptitle}\centering\huge}
  \posttitle{\par}
  \author{Avila Santos Alex}
  \preauthor{\centering\large\emph}
  \postauthor{\par}
  \predate{\centering\large\emph}
  \postdate{\par}
  \date{11 de junio de 2018}


\begin{document}
\maketitle

\subsubsection{Ejercicio 2 : En cada una de las siguientes líneas de
código, identifica qué estilo de coincidencia de argumentos se está
utilizando: exacto, parcial, posicional o mixto. Si es mixto, identifica
qué argumentos se especifican en cada
estilo.}\label{ejercicio-2-en-cada-una-de-las-siguientes-lineas-de-codigo-identifica-que-estilo-de-coincidencia-de-argumentos-se-esta-utilizando-exacto-parcial-posicional-o-mixto.-si-es-mixto-identifica-que-argumentos-se-especifican-en-cada-estilo.}

\paragraph{El entorno de evaluación de una
función}\label{el-entorno-de-evaluacion-de-una-funcion}

\paragraph{Cuando se llama o se invoca una función, se crea un nuevo
marco de evaluación. En este marco los argumentos formales se
corresponden con los argumentos suministrados de acuerdo con las reglas
de Argumento de argumentos (abajo). Las declaraciones en el cuerpo de la
función son evaluadas secuencialmente en este marco de
entorno.}\label{cuando-se-llama-o-se-invoca-una-funcion-se-crea-un-nuevo-marco-de-evaluacion.-en-este-marco-los-argumentos-formales-se-corresponden-con-los-argumentos-suministrados-de-acuerdo-con-las-reglas-de-argumento-de-argumentos-abajo.-las-declaraciones-en-el-cuerpo-de-la-funcion-son-evaluadas-secuencialmente-en-este-marco-de-entorno.}

\paragraph{Lo primero que ocurre en una evaluación de función es la
coincidencia de lo formal con el argumentos reales o suministrados. Esto
se hace mediante un proceso de tres
pasos:}\label{lo-primero-que-ocurre-en-una-evaluacion-de-funcion-es-la-coincidencia-de-lo-formal-con-el-argumentos-reales-o-suministrados.-esto-se-hace-mediante-un-proceso-de-tres-pasos}

. Exacta . Para cada argumento suministrado con nombre la lista de
argumentos formales se busca un artículo cuyo nombre coincida
exactamente.

. Parcial . Cada argumento suministrado con nombre se compara con el
resto argumentos formales que usan correspondencia parcial. Si el nombre
del argumento proporcionado coincide exactamente con la primera parte de
un argumento formal, entonces los dos argumentos se consideran para ser
emparejado.

. Posicional. Cada argumento suministrado no tiene que coincidir
exactamente sino se respeta el orden. Si hay un argumento, retomará los
argumentos restantes, etiquetado o no.

. Si algún argumento permanece sin coincidencia, se declara un error.

Los argumentos suministrados y los argumentos predeterminados se tratan
de manera diferente. Los argumentos suministrados a una función se
evalúan en el marco de evaluación de la función de llamada. El valor por
defecto los argumentos a una función se evalúan en el marco de
evaluación de la función. En general, los argumentos suministrados se
comportan como si fueran variables locales inicializadas con el valor
proporcionado y el nombre del argumento formal correspondiente. Cambiar
el valor de un suministram o argumento dentro de una función no afectará
el valor de la variable en el fram de llamada.

\begin{Shaded}
\begin{Highlighting}[]
\KeywordTok{array}\NormalTok{(}\DecValTok{8}\OperatorTok{:}\DecValTok{1}\NormalTok{,}\DataTypeTok{dim=}\KeywordTok{c}\NormalTok{(}\DecValTok{2}\NormalTok{,}\DecValTok{2}\NormalTok{,}\DecValTok{2}\NormalTok{))}
\end{Highlighting}
\end{Shaded}

\begin{verbatim}
## , , 1
## 
##      [,1] [,2]
## [1,]    8    6
## [2,]    7    5
## 
## , , 2
## 
##      [,1] [,2]
## [1,]    4    2
## [2,]    3    1
\end{verbatim}

\begin{Shaded}
\begin{Highlighting}[]
\CommentTok{#Argumentos mixto : posicional y exacto respectivamente}

\KeywordTok{rep}\NormalTok{(}\DecValTok{1}\OperatorTok{:}\DecValTok{2}\NormalTok{,}\DecValTok{3}\NormalTok{)}
\end{Highlighting}
\end{Shaded}

\begin{verbatim}
## [1] 1 2 1 2 1 2
\end{verbatim}

\begin{Shaded}
\begin{Highlighting}[]
\CommentTok{#Argumentos posicionales}

\KeywordTok{seq}\NormalTok{(}\DataTypeTok{from=}\DecValTok{10}\NormalTok{,}\DataTypeTok{to=}\DecValTok{8}\NormalTok{,}\DataTypeTok{length=}\DecValTok{5}\NormalTok{)}
\end{Highlighting}
\end{Shaded}

\begin{verbatim}
## [1] 10.0  9.5  9.0  8.5  8.0
\end{verbatim}

\begin{Shaded}
\begin{Highlighting}[]
\CommentTok{#Argumentos exactos}

\KeywordTok{sort}\NormalTok{(}\DataTypeTok{decreasing=}\NormalTok{T,}\DataTypeTok{x=}\KeywordTok{c}\NormalTok{(}\DecValTok{2}\NormalTok{,}\DecValTok{1}\NormalTok{,}\DecValTok{1}\NormalTok{,}\DecValTok{2}\NormalTok{,}\FloatTok{0.3}\NormalTok{,}\DecValTok{3}\NormalTok{,}\FloatTok{1.3}\NormalTok{))}
\end{Highlighting}
\end{Shaded}

\begin{verbatim}
## [1] 3.0 2.0 2.0 1.3 1.0 1.0 0.3
\end{verbatim}

\begin{Shaded}
\begin{Highlighting}[]
\CommentTok{#Argumentos exactos}

\KeywordTok{which}\NormalTok{(}\KeywordTok{matrix}\NormalTok{(}\KeywordTok{c}\NormalTok{(T,F,T,T),}\DecValTok{2}\NormalTok{,}\DecValTok{2}\NormalTok{))}
\end{Highlighting}
\end{Shaded}

\begin{verbatim}
## [1] 1 3 4
\end{verbatim}

\begin{Shaded}
\begin{Highlighting}[]
\CommentTok{#Argumentos posicional }

\KeywordTok{which}\NormalTok{(}\KeywordTok{matrix}\NormalTok{(}\KeywordTok{c}\NormalTok{(T,F,T,T),}\DecValTok{2}\NormalTok{,}\DecValTok{2}\NormalTok{),}\DataTypeTok{a=}\NormalTok{T)}
\end{Highlighting}
\end{Shaded}

\begin{verbatim}
##      row col
## [1,]   1   1
## [2,]   1   2
## [3,]   2   2
\end{verbatim}

\begin{Shaded}
\begin{Highlighting}[]
\CommentTok{#Argumento parcial}
\end{Highlighting}
\end{Shaded}


\end{document}
